\documentclass[a4paper,14pt]{extarticle}
\usepackage[T1,T2A]{fontenc}
\usepackage[utf8]{inputenc}
\usepackage[english,russian,ukrainian]{babel}
\usepackage{minted}
\usepackage{geometry}
\usepackage{graphicx}
\usepackage{fontspec} 
\usepackage{dirtytalk}
\usepackage{amsmath}
\usepackage{hyperref}

\def\equationautorefname~#1\null{(#1)\null}

\setmainfont{Liberation Serif}

\geometry{
    a4paper,
    left=20mm,
    top=20mm,
    right=10mm,
    bottom=20mm
}

\setlength{\emergencystretch}{2em}
\linespread{1.25}

\begin{document}
\begin{titlepage}
	\centering
    Міністерство освіти і науки України
    
    Харківський національний университет радіоелектроніки

    \vspace{1cm}
    Кафедра штучного інтелекту

    \vspace{2cm}
    Дисципліна: \say{Теорія прийняття рішень}

    \vspace{2cm}
    \uppercase{Лабораторна робота 5}

    
    % название лабораторной работы
    \uppercase{\say{РАНЖИРОВАНИЕ ОБЪЕКТОВ ПО РЕЗУЛЬТАТАМ
    ПОПАРНЫХ СРАВНЕНИЙ}}

    \vspace{4cm}
    \begin{minipage}[t]{10cm}
        Виконав ст. гр. ІТШІ-18-1:\\
        Соколенко Дмитро Олександрович
    \end{minipage}
    \hfill
    \begin{minipage}[t]{6cm}
        Прийняв:\\
        ст. в. Стьопін О. С.\\
        з оцінкою \say{\rule{2cm}{0.15mm}}\\
        \say{\rule{0.7cm}{0.15mm}}\rule{2cm}{0.15mm}20\rule{0.7cm}{0.15mm}р
    \end{minipage}

	\vfill

	{Харків \the\year{}}
\end{titlepage}
\section*{Вступ}
Изучение методов ранжирования объектов

\section{Аналіз предметної області}

\section{Порівняння до альтернатив}

\section{Технології та середа розробки}

\section{Опис програмного продукту}

\section*{Висновки}
Во время выполнения лабораторной работы мы 
изучили методы ранжирования объектов.

\end{document}


% \begin{figure}[H]
%     \centering
%     \includegraphics[width=0.75\textwidth]{rplot_iris.png}
%     \caption{Графік дерева вибірки Iris з дискретизацією через IG}
%     \label{fig:plot3}
% \end{figure}

% \begin{table}[H]
%     \centering
%     \begin{tabular}{ |c|c|c|c| } 
%      \hline
%      Мова   & \multicolumn{3}{c|}{Метод та датасет} \\ \hline
%             & Iris Thr. & Iris IG & SUSY \\ \hline
%      Python & 0.9333    & 0.9333  & 0.6799 \\ \hline
%      R      & 0.9333    & 0.9777  & 0.6594 \\ 
%      \hline
%     \end{tabular}
%     \caption{Порівняння точності класифікації}
%     \label{tab:t1}
% \end{table}
% \inputminted[breaklines,linenos=true]{scilab}{repl235.txt}