\documentclass{beamer}

\usepackage[utf8]{inputenc} % Set Encoding
\usepackage[T1]{fontenc} % Enable cyrillic fonts
\usepackage{fontspec} % Using custom fonts (requires -xelatex flag)
\usepackage{graphicx} % For image insertions
\usepackage{float} % For positioning
\usepackage[section]{placeins}
\usepackage{fontspec}

\setmainfont[
  Ligatures=TeX,
  Extension=.otf,
  BoldFont=cmunbx,
  ItalicFont=cmunti,
  BoldItalicFont=cmunbi,
]{cmunrm}
\setsansfont[
  Ligatures=TeX,
  Extension=.otf,
  BoldFont=cmunsx,
  ItalicFont=cmunsi,
]{cmunss}

\title{Бот Dvango}
\subtitle{Системи інтелектуальної обробки природо-мовної інформації}
\author[shortname]{Апраксін Антон Романович \\ \and Соколенко Дмитро Олександрович}
\institute{ІТШІ-18-1}
\date{Харків 2021}

\begin{document}
    \frame{\titlepage}

    \begin{frame}
        \frametitle{Вступ}
            Дана робота присвячена побудові чатботу, що відповідає на запитання користувача, які побудовані природною мовою. Бот не підтримує діалог на побутові теми, тобто йому не можна написати ``Hello'' або ``How are you?''. Але бот дуже гарно (як для бота) відповідає на запитання, що поставив користувач.
    \end{frame}

    \begin{frame}
        \frametitle{Актуальність}
    Ми живемо в епоху інформаційних технологій, які увійшли у життя кожного з нас. Кількість інформації зростає з кожним днем і знаходити відповіді на питання, що вас цікавлять стає складніше. З іншого ж боку, технології машинного навчання теж зростають так швидко, як ніяк раніше. І це призводить нас до концепту: використати сучасні технології для розв'язання сучасних проблем. Тобто ми можемо побудувати таку систему, що буде знаходити відповіді на питання, що нас цікавлять. І робити система це може інтерактивно, тобто у форматі чатботу, якому ви пишете повідомлення, яке містить питання і на яке бот якось змістовно відповідає.
    \end{frame}

    \begin{frame}
        \frametitle{Історія розвитку}
        Одним з перших віртуальних чатботів була програма ELIZA. В її основі лежить алгоритм, що перефразовував питання користувача, виділяючи в ньому незмінну частину. Бота можна було викрити за два запитання. Потім з'являлися більш сучасні та продвинуті моделі. З появою нейронних мереж потужність таких систем сильно зросла. Почали з'являтися системи, які можуть ввести в оману більше ніж 50\% журі. 

        Сьогодні віртуальних асистентів можна зустріти у більшості компаній. Вони допомагають користувачам вирішувати проблеми, що виникають на етапі користування системою.
    \end{frame}

    \begin{frame}
        \frametitle{Основні підходи до вирішення проблеми}
        Сьогодні існує багато алгоритмів, які допомагають будуватирозумні системи з обробки природо-мовної інформації. Це ісистеми побудовані на простих алгоритмах, таких як наївнийБайес або програма ELIZA. Але також існують сучаснінейромережі, в яких набагато більний потенціал. Архітектурасучаної нейромережі, що часто використовується –трансформер. Саме цю архітектуру ми і використовували.
    \end{frame}

    \begin{frame}
        \frametitle{Відомі проекти}
        \begin{enumerate}
            \item ELIZA - перший більш-менш успішний проект
            \item PARRY - з'явився трохи пізніше ELIZA. Пародувала людину, яка намагався змоделювати поведінку параноїдального шизофреніка.
            \item CleverBot - пройшов 59\% тестів Тьюринга в 2011 році.
        \end{enumerate}
    \end{frame}

    \begin{frame}
        \frametitle{Перспективи розвитку}
        Дана галузь розвивається дуже швидко. Є як академічний так і економічний інтерес. Науковці бажають першими пройти успішно тест Тьринга, компанії намагаються як найбільше все автоматизувати, тому теж інвестують в цю галузь.
    \end{frame}

\end{document}
