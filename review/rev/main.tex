\documentclass[a4paper,14pt]{extarticle}
\usepackage[T1,T2A]{fontenc}
\usepackage[utf8]{inputenc}
\usepackage[english,russian,ukrainian]{babel}
\usepackage{minted}
\usepackage{geometry}
\usepackage{graphicx}
\usepackage{fontspec}
\usepackage{dirtytalk}
\usepackage{amsmath}
\usepackage{hyperref}
\usepackage{indentfirst}

\def\equationautorefname~#1\null{(#1)\null}

\setmainfont{Liberation Serif}

\geometry{
    a4paper,
    left=20mm,
    top=20mm,
    right=10mm,
    bottom=20mm
}

\setlength{\emergencystretch}{2em}
\linespread{1.25}

\begin{document}
\begin{titlepage}
	\centering
    Міністерство освіти і науки України
    
    Харківський національний университет радіоелектроніки

    \vspace{1cm}
    Кафедра штучного інтелекту

    \vspace{2cm}
    Дисципліна: \say{Теорія прийняття рішень}

    \vspace{2cm}
    \uppercase{Лабораторна робота 5}

    
    % название лабораторной работы
    \uppercase{\say{РАНЖИРОВАНИЕ ОБЪЕКТОВ ПО РЕЗУЛЬТАТАМ
    ПОПАРНЫХ СРАВНЕНИЙ}}

    \vspace{4cm}
    \begin{minipage}[t]{10cm}
        Виконав ст. гр. ІТШІ-18-1:\\
        Соколенко Дмитро Олександрович
    \end{minipage}
    \hfill
    \begin{minipage}[t]{6cm}
        Прийняв:\\
        ст. в. Стьопін О. С.\\
        з оцінкою \say{\rule{2cm}{0.15mm}}\\
        \say{\rule{0.7cm}{0.15mm}}\rule{2cm}{0.15mm}20\rule{0.7cm}{0.15mm}р
    \end{minipage}

	\vfill

	{Харків \the\year{}}
\end{titlepage}
\section*{Вступ}
Изучение методов ранжирования объектов

\section{Аналіз предметної області}

\section{Порівняння до альтернатив}

\section{Технології та середа розробки}
\subsection{Мова програмування}
Для написання роботи була використана мова програмування
Python. Python -- інтерпретована мова програмування загального
призначення високого рівня. Філософія дизайну Python наголошує
на читабельності коду завдяки помітному використанню значних
відступів. Його мовні конструкції, а також об’єктно-орієнтований
підхід мають на меті допомогти програмістам писати чіткий
логічний код для малих та великих проектів.

Python зосереджується на читабельності коду. Мова універсальна,
акуратна, проста у використанні та вивченні, читабельна
та добре структурована.

Завдяки гнучкості Python легко провести дослідницький аналіз даних.
Також він дозволяє використовувати найкращі з різних парадигм
програмування. Він об’єктно-орієнтований, але також має функції
з функціонального програмування.

З переваг цієї мови можна виділити:
\begin{enumerate}
    \item Відкритий код.

    Інтерпртатор Python можна завантажити безкоштовно та переглянути
    його вихідний код. Це також означає, що мова розвивається згідно
    до потреб суспільства програмістів, та не може раптово змінити
    курс розвитку.

    \item Проста мова.

    Python можна швидко вивчити, та відразу розпочати розробляти програми.

    \item Бібліотеки.

    Для Python написано дуже багато бібліотек для різних задач, тому
    можна відразу розпочати вирішування поточної задачі
    не турбуючись про інфраструктуру.
    Також мова програмування поставляється з менеджером пакетів Pip,
    який дозволяє завантажувати всі бібліотеки, які можуть
    знадобитися, однією командою.
\end{enumerate}

Головні недоліки мови:
\begin{enumerate}
    \item Повільна швидкість.

    Python -- це інтерпретована мова, тому вона працює повільніше,
    ніж деякі інші популярні мови програмування.

    \item Споживання пам'яті.

    Python робить компроміс заради простоти, тому споживання
    пам'яті у нього велике. Це може бути проблемою для задач, що
    потребують багато одночасно активних об'єктів в пам'яті,
    але не для нашої.
\end{enumerate}

Виходячи з приведених переваг та недоліків, ми вирішили, що
Python найбільш гарно підходить для виконання даної роботи.

\subsection{Середа розробки}


\subsection{Мова розмітки для написання звіту}


\section{}

\section{Опис програмного продукту}

\section*{Висновки}
Во время выполнения лабораторной работы мы 
изучили методы ранжирования объектов.

\end{document}


% \begin{figure}[H]
%     \centering
%     \includegraphics[width=0.75\textwidth]{rplot_iris.png}
%     \caption{Графік дерева вибірки Iris з дискретизацією через IG}
%     \label{fig:plot3}
% \end{figure}

% \begin{table}[H]
%     \centering
%     \begin{tabular}{ |c|c|c|c| } 
%      \hline
%      Мова   & \multicolumn{3}{c|}{Метод та датасет} \\ \hline
%             & Iris Thr. & Iris IG & SUSY \\ \hline
%      Python & 0.9333    & 0.9333  & 0.6799 \\ \hline
%      R      & 0.9333    & 0.9777  & 0.6594 \\ 
%      \hline
%     \end{tabular}
%     \caption{Порівняння точності класифікації}
%     \label{tab:t1}
% \end{table}
% \inputminted[breaklines,linenos=true]{scilab}{repl235.txt}